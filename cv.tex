\documentclass{cv}
\usepackage{hyperref}
\usepackage{enumitem}
 
\begin{document}
 
\name{Anirudhan Jegannathan Rajagopalan}
\addressline{27C/2, Muniasamipuram, II Street,}
\addressline{Tuticorin --- 628003}
\addressline{+91--8106032060}
\addressline{http://rajegannathan.in}
\addressline{anirudhan@rajegannathan.in}

\section{Education}
\begin{description}[leftmargin=55pt,labelwidth=50pt]
  \item[B.Tech] Information Technology.\hfill\textit{2006--2010}
    \\Government College of Technology, Coimbatore.
    \\\textit{CGPA}:7.99/10.  \textbf{First class with Distinction.}
  \item[HSC] Tamil Nadu State Board.\hfill\textit{2006}
    \\Sakthi Vinayakar Hindu Vidyalaya, Tuticorin.
    \\\textit{Score}: 1091/1200. (90.91\%) \textbf{Third rank in school.}
  \item[AISSE] Central Board of Secondary Examinations.\hfill{2004}
    \\Sakthi Vinayakar Hindu Vidyalaya, Tuticorin.
    \\\textit{Score}: 422/500. (84.4\%)
\end{description}
 
\section{Awards/Honour}
\begin{itemize}[label={$\ast$}]
  \item \textbf{First prize in Insideview Hackaton.} \hfill Aug--2012\\
    Learnt \& built a \textit{Company Recommendation Engine} using \textit{N-Nearest Neighbour} algorithm.  Won first prize out of 25+ teams.
  \item \textbf{First prize in Project Presentation at Srishti 09, PSG Tech.} \hfill Feb--2009\\
    Awarded first prize for \textit{Real-time Desktop search application for linux}. Srishti is an annual symposium conducted by the IEEE students' chapter of PSG Tech, Coimbatore.  It attracts more than 3000 participants from all over India.
  \item \textbf{Second prize in Project Presentation at Soft Sem Quiz.}\hfill Dec--2007\\
    Awarded second prize for completing \textit{SMS Simulator app with T9 dictionary} and \textit{Parser} for generating algebric expression from given Infix and Postfix/prefix expressions.  Soft Sem Quiz is the annual intra college technical event.
  \item \textbf{First prize in district level Mathematics olympiad.} \hfill Oct--2005\\
    Mathematics quiz on \textit{Calculas, Geometry, Trignometry and on other Higher Secondary syllabus}.  A total of 22 teams from Tuticorin District participated in the event.
  \item \textbf{Top 10\% scorers of my Institution in Junior level.} \hfill 2003\\
    Awared by \textit{The Association of Mathematics Teachers of India} for my commendable performance in Junior level screening test for Mathematics Olympiad.
\end{itemize}

\section{Undergraduate Project}
\begin{description}[leftmargin=50pt,labelwidth=50pt]
  \item[Title] An efficient key management protocol with robust continuity for wireless sensor networks.
  \item[Guide] Dr.\ K. Baskaran, M.E. Ph.D.
  \item[] Wireless Sensor Networks (WSN) are vulnerable to various types of attacks.  A big issue in security schemes is the key management mechanism responsible for distributing secret keys.  The wireless sensor networks use pairwise key management protocols.  The traditional pairwise key management protocols use limited and fixed pool size for storing the keys.  If nodes are compromised, the keys must be regenreated every time.  The existing security schemes for WSNs use Blom's key approach for key management which results in time complexity and fixed pool size.\\
    A modified version of the Blom's scheme for efficient key management in wireless sensor networks is proposed in this project.  This scheme provides more security by using a hashing-like algorithm along with modified Blom's key approach for securing the secret key information with scalable key pool size.
\end{description}

\section{Academic/Hobby Projects}
\begin{description}[leftmargin=50pt,labelwidth=50pt]
  \item[Gre Wordcards] Android application for helping my GRE Vocabulary preparation.
    \begin{itemize}[label={},leftmargin=10pt,topsep=0pt]
      \item[\textbf{--}] Github Url: \url{https://github.com/rajegannathan/custom-vocab-flashcards.git}
      \item[\textbf{--}] It uses \textit{Wordnik Words API} to fetch Meaning, Usage, Etymology and Derivatives of any given word.
      \item[\textbf{--}] Opens up in full screen to avoid any distraction from other notifications.  Uses big \& bold font for helping the student read easily.
      \item[\textbf{--}] Uses double tap \& swipe gestures to go between words and is optimized for single hand usage (which is my majority use case).
      \item[\textbf{--}] Uses background threads for fetching the details from API and cahces around 250 latest words added.
    \end{itemize}
  \item[infoquestgct.com] The official website of Infoquest 2010.  
    \begin{itemize}[label={},leftmargin=10pt,topsep=0pt]
      \item[\textbf{--}] Infoquest is a National level technical symposium conducted by Computer Science \& Information Technology Association of Government College of Technology, Coimbatore.
      \item[\textbf{--}] Designed \& Developed \textit{User authentication, Custom Captcha, Virtual Stock Market, Online quiz and Online Programming} modules in PHP\@.
      \item[\textbf{--}] Responsible for the whole backend design \& development.
    \end{itemize}
  \item[Online Judge] A judge similar to SPOJ\@.
    \begin{itemize}[label={},leftmargin=10pt,topsep=0pt]
      \item[\textbf{--}] The judge was used for compiling programming solutions submitted for the Online Programming contest of Infoquest 2010.
      \item[\textbf{--}] Supported soulutions in five languages. Perl, Java, C, C++, Ruby.
      \item[\textbf{--}] Supported facility to set \textit{Memory, Process, File, and CPU-time} limits per question.
      \item[\textbf{--}] Designed as three separate process which communicates using \textit{Unix Pipes}.  First module polls and downloads submitted programs.  Second module compiles the program with given run time limits and checks the output for correctness.  Thrid module uploads the result back to remote server for display in web interface.
    \end{itemize}
  \item[Desktop Search] A realtime desktop search application for Fedora Linux.
    \begin{itemize}[label={},leftmargin=10pt,topsep=0pt]
      \item[\textbf{--}] A GUI application for linux developed entirely in C++.  The GUI was built with \textit{GTK tooklit}.
      \item[\textbf{--}] The applicaiton indexes the whole file system as a \textit{n-ary tree}.
      \item[\textbf{--}] It uses linux \textit{watch descriptors} to watch on any filesystem changes and updates the index.
      \item[\textbf{--}] The application could match search terms which match more than \textit{200,000} entries in 2 seconds.  The index contained a total of 1 Million entries.
    \end{itemize}
  \item[SMS Simulator] A LAN Chat application with mobile like interface and T9 dictionary.
    \begin{itemize}[label={},leftmargin=10pt,topsep=0pt]
      \item[\textbf{--}] The application mimics mobile networks by having a centralized server and a number of clients that communicate messages through the server using TCP\@.
      \item[\textbf{--}] Has a mobile like Tactile keyboard for input.  Supports auto-complete similar to T9 mobile keyboards.  Words can be added/removed to the dictionary.
      \item[\textbf{--}] The dictionary is implemented using \textit{Trie} datastructure.
      \item[\textbf{--}] Uses AWT for the UI and Java for the server side programming.
    \end{itemize}
\end{description}

\end{document}
